% Options for packages loaded elsewhere
\PassOptionsToPackage{unicode}{hyperref}
\PassOptionsToPackage{hyphens}{url}
%
\documentclass[
]{article}
\usepackage{amsmath,amssymb}
\usepackage{iftex}
\ifPDFTeX
  \usepackage[T1]{fontenc}
  \usepackage[utf8]{inputenc}
  \usepackage{textcomp} % provide euro and other symbols
\else % if luatex or xetex
  \usepackage{unicode-math} % this also loads fontspec
  \defaultfontfeatures{Scale=MatchLowercase}
  \defaultfontfeatures[\rmfamily]{Ligatures=TeX,Scale=1}
\fi
\usepackage{lmodern}
\ifPDFTeX\else
  % xetex/luatex font selection
\fi
% Use upquote if available, for straight quotes in verbatim environments
\IfFileExists{upquote.sty}{\usepackage{upquote}}{}
\IfFileExists{microtype.sty}{% use microtype if available
  \usepackage[]{microtype}
  \UseMicrotypeSet[protrusion]{basicmath} % disable protrusion for tt fonts
}{}
\makeatletter
\@ifundefined{KOMAClassName}{% if non-KOMA class
  \IfFileExists{parskip.sty}{%
    \usepackage{parskip}
  }{% else
    \setlength{\parindent}{0pt}
    \setlength{\parskip}{6pt plus 2pt minus 1pt}}
}{% if KOMA class
  \KOMAoptions{parskip=half}}
\makeatother
\usepackage{xcolor}
\usepackage[margin=1in]{geometry}
\usepackage{color}
\usepackage{fancyvrb}
\newcommand{\VerbBar}{|}
\newcommand{\VERB}{\Verb[commandchars=\\\{\}]}
\DefineVerbatimEnvironment{Highlighting}{Verbatim}{commandchars=\\\{\}}
% Add ',fontsize=\small' for more characters per line
\usepackage{framed}
\definecolor{shadecolor}{RGB}{248,248,248}
\newenvironment{Shaded}{\begin{snugshade}}{\end{snugshade}}
\newcommand{\AlertTok}[1]{\textcolor[rgb]{0.94,0.16,0.16}{#1}}
\newcommand{\AnnotationTok}[1]{\textcolor[rgb]{0.56,0.35,0.01}{\textbf{\textit{#1}}}}
\newcommand{\AttributeTok}[1]{\textcolor[rgb]{0.13,0.29,0.53}{#1}}
\newcommand{\BaseNTok}[1]{\textcolor[rgb]{0.00,0.00,0.81}{#1}}
\newcommand{\BuiltInTok}[1]{#1}
\newcommand{\CharTok}[1]{\textcolor[rgb]{0.31,0.60,0.02}{#1}}
\newcommand{\CommentTok}[1]{\textcolor[rgb]{0.56,0.35,0.01}{\textit{#1}}}
\newcommand{\CommentVarTok}[1]{\textcolor[rgb]{0.56,0.35,0.01}{\textbf{\textit{#1}}}}
\newcommand{\ConstantTok}[1]{\textcolor[rgb]{0.56,0.35,0.01}{#1}}
\newcommand{\ControlFlowTok}[1]{\textcolor[rgb]{0.13,0.29,0.53}{\textbf{#1}}}
\newcommand{\DataTypeTok}[1]{\textcolor[rgb]{0.13,0.29,0.53}{#1}}
\newcommand{\DecValTok}[1]{\textcolor[rgb]{0.00,0.00,0.81}{#1}}
\newcommand{\DocumentationTok}[1]{\textcolor[rgb]{0.56,0.35,0.01}{\textbf{\textit{#1}}}}
\newcommand{\ErrorTok}[1]{\textcolor[rgb]{0.64,0.00,0.00}{\textbf{#1}}}
\newcommand{\ExtensionTok}[1]{#1}
\newcommand{\FloatTok}[1]{\textcolor[rgb]{0.00,0.00,0.81}{#1}}
\newcommand{\FunctionTok}[1]{\textcolor[rgb]{0.13,0.29,0.53}{\textbf{#1}}}
\newcommand{\ImportTok}[1]{#1}
\newcommand{\InformationTok}[1]{\textcolor[rgb]{0.56,0.35,0.01}{\textbf{\textit{#1}}}}
\newcommand{\KeywordTok}[1]{\textcolor[rgb]{0.13,0.29,0.53}{\textbf{#1}}}
\newcommand{\NormalTok}[1]{#1}
\newcommand{\OperatorTok}[1]{\textcolor[rgb]{0.81,0.36,0.00}{\textbf{#1}}}
\newcommand{\OtherTok}[1]{\textcolor[rgb]{0.56,0.35,0.01}{#1}}
\newcommand{\PreprocessorTok}[1]{\textcolor[rgb]{0.56,0.35,0.01}{\textit{#1}}}
\newcommand{\RegionMarkerTok}[1]{#1}
\newcommand{\SpecialCharTok}[1]{\textcolor[rgb]{0.81,0.36,0.00}{\textbf{#1}}}
\newcommand{\SpecialStringTok}[1]{\textcolor[rgb]{0.31,0.60,0.02}{#1}}
\newcommand{\StringTok}[1]{\textcolor[rgb]{0.31,0.60,0.02}{#1}}
\newcommand{\VariableTok}[1]{\textcolor[rgb]{0.00,0.00,0.00}{#1}}
\newcommand{\VerbatimStringTok}[1]{\textcolor[rgb]{0.31,0.60,0.02}{#1}}
\newcommand{\WarningTok}[1]{\textcolor[rgb]{0.56,0.35,0.01}{\textbf{\textit{#1}}}}
\usepackage{graphicx}
\makeatletter
\def\maxwidth{\ifdim\Gin@nat@width>\linewidth\linewidth\else\Gin@nat@width\fi}
\def\maxheight{\ifdim\Gin@nat@height>\textheight\textheight\else\Gin@nat@height\fi}
\makeatother
% Scale images if necessary, so that they will not overflow the page
% margins by default, and it is still possible to overwrite the defaults
% using explicit options in \includegraphics[width, height, ...]{}
\setkeys{Gin}{width=\maxwidth,height=\maxheight,keepaspectratio}
% Set default figure placement to htbp
\makeatletter
\def\fps@figure{htbp}
\makeatother
\setlength{\emergencystretch}{3em} % prevent overfull lines
\providecommand{\tightlist}{%
  \setlength{\itemsep}{0pt}\setlength{\parskip}{0pt}}
\setcounter{secnumdepth}{-\maxdimen} % remove section numbering
\ifLuaTeX
  \usepackage{selnolig}  % disable illegal ligatures
\fi
\usepackage{bookmark}
\IfFileExists{xurl.sty}{\usepackage{xurl}}{} % add URL line breaks if available
\urlstyle{same}
\hypersetup{
  pdftitle={Mémoire brouillon},
  hidelinks,
  pdfcreator={LaTeX via pandoc}}

\title{Mémoire brouillon}
\author{}
\date{\vspace{-2.5em}2025-01-29}

\begin{document}
\maketitle

rm(list = ls())

\subsection{3 Two samples hypothesis
testing}\label{two-samples-hypothesis-testing}

\section{3.1 Global likelihood ratio
test}\label{global-likelihood-ratio-test}

\section{Pour calculer la Log-Vraisemblance
:}\label{pour-calculer-la-log-vraisemblance}

-Structure conforme à l'équation (4) -Utilistion de dgamma et dweibull
pour les lois paramétrique

les tests sont développés pour des lois à 2 paramètres (Gamma/weibull),
pourquoi ? dans la section 3.2 lors du calcul des degrés de liberté pour
le test asymptotique : \textbar w\textbar=2. De plus dans la section 5.1
et 6.2.2 pour la simulation on considère ces deux distributions. Les
équations (2) et (4) supposent une parmétrisation à 2 variables pour les
MLE ce choix est pertinent par : la flexibilité de ces lois pour
modéliser des durées non exponentielles, leur adéquation avec les
asymptotiques du test du rapport de vraissemblance.

\begin{Shaded}
\begin{Highlighting}[]
\CommentTok{\#Fonction qui calcule la log{-}vraisemblance d\textquotesingle{}un ensemble de trajectoires d\textquotesingle{}un SMP.}
\CommentTok{\#Params : contient alpha : les proba initiales des états, P la matrice de transition entre les états, omega la liste des paramètres des lois de durée associées aux états}
\CommentTok{\#trajectories : liste avec les trajectoires observées}
\CommentTok{\#dist\_type : type de distribution pour les temps de séjour, ici prend que gamma ou Weibull.}
\NormalTok{log\_likelihood }\OtherTok{\textless{}{-}} \ControlFlowTok{function}\NormalTok{(params, trajectories, absorbing\_state, dist\_type) \{}
\NormalTok{  alpha }\OtherTok{\textless{}{-}}\NormalTok{ params}\SpecialCharTok{$}\NormalTok{alpha}
\NormalTok{  P }\OtherTok{\textless{}{-}}\NormalTok{ params}\SpecialCharTok{$}\NormalTok{P}
\NormalTok{  omega }\OtherTok{\textless{}{-}}\NormalTok{ params}\SpecialCharTok{$}\NormalTok{omega }\CommentTok{\#On extrait les données depuis params}
\NormalTok{  logL }\OtherTok{\textless{}{-}} \DecValTok{0} \CommentTok{\# Initialisation de la log vraisemblance}
  
  \ControlFlowTok{for}\NormalTok{ (traj }\ControlFlowTok{in}\NormalTok{ trajectories) \{ }\CommentTok{\# On parcourt toutes les trajectoires}
\NormalTok{    states }\OtherTok{\textless{}{-}}\NormalTok{ traj}\SpecialCharTok{$}\NormalTok{states }\CommentTok{\#ordre des états}
\NormalTok{    times }\OtherTok{\textless{}{-}}\NormalTok{ traj}\SpecialCharTok{$}\NormalTok{times }\CommentTok{\#durées passées dans chaque état}
    
    \CommentTok{\# Terme initial alpha\_j1}
\NormalTok{    logL }\OtherTok{\textless{}{-}}\NormalTok{ logL }\SpecialCharTok{+} \FunctionTok{log}\NormalTok{(alpha[states[}\DecValTok{1}\NormalTok{]]) }\CommentTok{\#ajoute la proba initiale de commencer dans un état s1 est alpha\_j1}
    
    \ControlFlowTok{for}\NormalTok{ (i }\ControlFlowTok{in} \DecValTok{1}\SpecialCharTok{:}\NormalTok{(}\FunctionTok{length}\NormalTok{(states)}\SpecialCharTok{{-}}\DecValTok{1}\NormalTok{)) \{}\CommentTok{\#Boucle sur les transitions entre états}
\NormalTok{      from }\OtherTok{\textless{}{-}}\NormalTok{ states[i] }\CommentTok{\# Etat de départ}
\NormalTok{      to }\OtherTok{\textless{}{-}}\NormalTok{ states[i}\SpecialCharTok{+}\DecValTok{1}\NormalTok{] }\CommentTok{\# Etat d\textquotesingle{}arrivée}
      
      \CommentTok{\# Terme de transition P\_ij}
\NormalTok{      logL }\OtherTok{\textless{}{-}}\NormalTok{ logL }\SpecialCharTok{+} \FunctionTok{log}\NormalTok{(P[from, to]) }\CommentTok{\#ajoute la proba de transition entre les états from et to}
      
      \CommentTok{\# Terme de durée (sauf dernière transition absorbante)}
      \ControlFlowTok{if}\NormalTok{ (to }\SpecialCharTok{!=}\NormalTok{ absorbing\_state) \{}\CommentTok{\#Si to est un état absorbant on passe à la transition suivante}
        \ControlFlowTok{if}\NormalTok{ (dist\_type }\SpecialCharTok{==} \StringTok{"gamma"}\NormalTok{) \{}\CommentTok{\#Cas ou la loi des durées suit une distribution gamma.}
\NormalTok{          logL }\OtherTok{\textless{}{-}}\NormalTok{ logL }\SpecialCharTok{+} \FunctionTok{dgamma}\NormalTok{(times[i],}\AttributeTok{shape=}\NormalTok{omega[[from]][}\DecValTok{1}\NormalTok{],}\AttributeTok{rate=}\NormalTok{omega[[from]][}\DecValTok{2}\NormalTok{], }\AttributeTok{log=}\ConstantTok{TRUE}\NormalTok{)}\CommentTok{\#Shape = paramètre k, rate=parametre teta}
\NormalTok{        \} }\ControlFlowTok{else} \ControlFlowTok{if}\NormalTok{ (dist\_type }\SpecialCharTok{==} \StringTok{"weibull"}\NormalTok{) \{}\CommentTok{\#Cas ou la loi des durées suit une distribution Weibull}
\NormalTok{          logL }\OtherTok{\textless{}{-}}\NormalTok{ logL }\SpecialCharTok{+} \FunctionTok{dweibull}\NormalTok{(times[i],}\AttributeTok{shape=}\NormalTok{omega[[from]][}\DecValTok{1}\NormalTok{],}\AttributeTok{scale=}\NormalTok{omega[[from]][}\DecValTok{2}\NormalTok{], }\AttributeTok{log=}\ConstantTok{TRUE}\NormalTok{)}
\NormalTok{        \} }\ControlFlowTok{else} \ControlFlowTok{if}\NormalTok{ (dist\_type }\SpecialCharTok{==} \StringTok{"exponential"}\NormalTok{) \{}
\NormalTok{          logL }\OtherTok{\textless{}{-}}\NormalTok{ logL }\SpecialCharTok{+} \FunctionTok{dexp}\NormalTok{(times[i], }\AttributeTok{rate =}\NormalTok{ omega[[from]][}\DecValTok{1}\NormalTok{], }\AttributeTok{log =} \ConstantTok{TRUE}\NormalTok{)}
\NormalTok{          \}}
\NormalTok{      \}}
\NormalTok{    \}}
\NormalTok{  \}}
  \FunctionTok{return}\NormalTok{(logL)}
\NormalTok{\}}
\end{Highlighting}
\end{Shaded}

\subsection{Estimation des
paramètres}\label{estimation-des-paramuxe8tres}

méthode : Maximum de vraisemblance Differentes méthodes pour
l'estimation des paramètres du SMP : optimx : rapide test differentes
méthodes,pas besoin de gerer les contraintes Newton Raphson : `nlm()',
convergence rapide mais peut échouer si la log-vraisemblance est trop
complexe

\begin{Shaded}
\begin{Highlighting}[]
\NormalTok{estimate\_SMP\_params }\OtherTok{\textless{}{-}} \ControlFlowTok{function}\NormalTok{(trajectories, absorbing\_state, dist\_type) \{}
  \CommentTok{\# Estimation α (fréquences initiales)}
\NormalTok{  initial\_states }\OtherTok{\textless{}{-}} \FunctionTok{sapply}\NormalTok{(trajectories, }\ControlFlowTok{function}\NormalTok{(x) x}\SpecialCharTok{$}\NormalTok{states[}\DecValTok{1}\NormalTok{]) }\CommentTok{\#Extrait le premier état de chaque trajectoire pour identifier les états initiaux}
\NormalTok{  alpha }\OtherTok{\textless{}{-}} \FunctionTok{table}\NormalTok{(}\FunctionTok{factor}\NormalTok{(initial\_states, }\AttributeTok{levels=}\DecValTok{1}\SpecialCharTok{:}\NormalTok{n\_states))}
\NormalTok{  alpha }\OtherTok{\textless{}{-}}\NormalTok{ alpha}\SpecialCharTok{/}\FunctionTok{sum}\NormalTok{(alpha) }\CommentTok{\# proportion de chaque état initial}
  
  \CommentTok{\# Estimation P (matrice de transitions)}
\NormalTok{  transition\_counts }\OtherTok{\textless{}{-}} \FunctionTok{matrix}\NormalTok{(}\DecValTok{0}\NormalTok{, n\_states, n\_states) }\CommentTok{\# Création matrice n*n}
  \ControlFlowTok{for}\NormalTok{ (traj }\ControlFlowTok{in}\NormalTok{ trajectories) \{}
\NormalTok{    states }\OtherTok{\textless{}{-}}\NormalTok{ traj}\SpecialCharTok{$}\NormalTok{states}
    \ControlFlowTok{for}\NormalTok{ (i }\ControlFlowTok{in} \DecValTok{1}\SpecialCharTok{:}\NormalTok{(}\FunctionTok{length}\NormalTok{(states)}\SpecialCharTok{{-}}\DecValTok{1}\NormalTok{)) \{}
\NormalTok{      transition\_counts[states[i], states[i}\SpecialCharTok{+}\DecValTok{1}\NormalTok{]] }\OtherTok{\textless{}{-}}\NormalTok{ transition\_counts[states[i], states[i}\SpecialCharTok{+}\DecValTok{1}\NormalTok{]] }\SpecialCharTok{+} \DecValTok{1}
\NormalTok{    \}}
\NormalTok{  \}}
\NormalTok{  P }\OtherTok{\textless{}{-}}\NormalTok{ transition\_counts}\SpecialCharTok{/}\FunctionTok{rowSums}\NormalTok{(transition\_counts)}
  
  \CommentTok{\# Estimation ω (MLE par état)}
\NormalTok{  omega }\OtherTok{\textless{}{-}} \FunctionTok{list}\NormalTok{()}
  \ControlFlowTok{for}\NormalTok{ (state }\ControlFlowTok{in} \DecValTok{1}\SpecialCharTok{:}\NormalTok{n\_states) \{}
\NormalTok{    durations }\OtherTok{\textless{}{-}} \FunctionTok{unlist}\NormalTok{(}\FunctionTok{lapply}\NormalTok{(trajectories, }\ControlFlowTok{function}\NormalTok{(traj) \{}
\NormalTok{      idx }\OtherTok{\textless{}{-}} \FunctionTok{which}\NormalTok{(traj}\SpecialCharTok{$}\NormalTok{states[}\SpecialCharTok{{-}}\FunctionTok{length}\NormalTok{(traj}\SpecialCharTok{$}\NormalTok{states)] }\SpecialCharTok{==}\NormalTok{ state) }\CommentTok{\#identifie les indices (idx) où l\textquotesingle{}état courant apparaît avant la dernière transition (on exclut l\textquotesingle{}état absorbant)}
      \ControlFlowTok{if}\NormalTok{(}\FunctionTok{length}\NormalTok{(idx) }\SpecialCharTok{\textgreater{}} \DecValTok{0}\NormalTok{) }\FunctionTok{return}\NormalTok{(traj}\SpecialCharTok{$}\NormalTok{times[idx]) }
      \ControlFlowTok{else} \FunctionTok{return}\NormalTok{(}\ConstantTok{NULL}\NormalTok{)}
\NormalTok{    \}))}
    
    \ControlFlowTok{if}\NormalTok{(dist\_type }\SpecialCharTok{==} \StringTok{"gamma"}\NormalTok{) \{}
\NormalTok{      fit }\OtherTok{\textless{}{-}} \FunctionTok{fitdistr}\NormalTok{(durations, }\StringTok{"gamma"}\NormalTok{, }\AttributeTok{start=}\FunctionTok{list}\NormalTok{(}\AttributeTok{shape=}\DecValTok{1}\NormalTok{, }\AttributeTok{rate=}\DecValTok{1}\NormalTok{))}
\NormalTok{      omega[[state]] }\OtherTok{\textless{}{-}} \FunctionTok{c}\NormalTok{(fit}\SpecialCharTok{$}\NormalTok{estimate[}\StringTok{"shape"}\NormalTok{], fit}\SpecialCharTok{$}\NormalTok{estimate[}\StringTok{"rate"}\NormalTok{])}
\NormalTok{    \} }\ControlFlowTok{else} \ControlFlowTok{if}\NormalTok{(dis\_type }\SpecialCharTok{==} \StringTok{"weibull"}\NormalTok{) \{}
\NormalTok{      fit }\OtherTok{\textless{}{-}} \FunctionTok{fitdistr}\NormalTok{(durations, }\StringTok{"weibull"}\NormalTok{, }\AttributeTok{start=}\FunctionTok{list}\NormalTok{(}\AttributeTok{shape=}\DecValTok{1}\NormalTok{, }\AttributeTok{scale=}\DecValTok{1}\NormalTok{))}
\NormalTok{      omega[[state]] }\OtherTok{\textless{}{-}} \FunctionTok{c}\NormalTok{(fit}\SpecialCharTok{$}\NormalTok{estimate[}\StringTok{"shape"}\NormalTok{], fit}\SpecialCharTok{$}\NormalTok{estimate[}\StringTok{"scale"}\NormalTok{]) }\CommentTok{\#La fonction fitdistr ajuste les paramètres de la distribution en maximisant la vraisemblance des données de durées observées.}
\NormalTok{    \} }\ControlFlowTok{else}\NormalTok{ \{}
\NormalTok{      fit }\OtherTok{\textless{}{-}} \FunctionTok{fitdistr}\NormalTok{(durations, }\StringTok{"exponential"}\NormalTok{, }\AttributeTok{start=}\FunctionTok{list}\NormalTok{(}\AttributeTok{shape=}\DecValTok{1}\NormalTok{))}
\NormalTok{      omega[[state]] }\OtherTok{\textless{}{-}} \FunctionTok{c}\NormalTok{(fit}\SpecialCharTok{$}\NormalTok{estimate[}\StringTok{"rate"}\NormalTok{])}
\NormalTok{    \}}
\NormalTok{  \}}
  \FunctionTok{return}\NormalTok{(}\FunctionTok{list}\NormalTok{(}\AttributeTok{alpha=}\NormalTok{alpha, }\AttributeTok{P=}\NormalTok{P, }\AttributeTok{omega=}\NormalTok{omega))}
\NormalTok{\}}
\end{Highlighting}
\end{Shaded}

Ici, sous H0 on combine trajectories1 et trajectories2 en un seul
ensemble avec c(trajectories1, trajectories2), car H0 suppose que les
deux ensembles suivent le même modèle.

\begin{Shaded}
\begin{Highlighting}[]
\CommentTok{\# trajectories1 : Premier ensemble de trajectoires.}
\CommentTok{\# trajectories2 : Deuxième ensemble de trajectoires.}
\CommentTok{\# absorbing\_state : L\textquotesingle{}état absorbant de la chaîne SMP.}
\CommentTok{\# dist\_type : Le type de distribution pour les durées}


\NormalTok{compute\_LR }\OtherTok{\textless{}{-}} \ControlFlowTok{function}\NormalTok{(trajectories1, trajectories2, absorbing\_state, dist\_type) \{}
  \CommentTok{\# Estimation sous H0 (données regroupées)}
\NormalTok{  mle\_H0 }\OtherTok{\textless{}{-}} \FunctionTok{estimate\_SMP\_params}\NormalTok{(}\FunctionTok{c}\NormalTok{(trajectories1, trajectories2), }
\NormalTok{                               absorbing\_state, dist\_type)}
  
  \CommentTok{\# Estimation sous H1 (données séparées)}
\NormalTok{  mle\_H1\_t1 }\OtherTok{\textless{}{-}} \FunctionTok{estimate\_SMP\_params}\NormalTok{(trajectories1, absorbing\_state, dist\_type)}
\NormalTok{  mle\_H1\_t2 }\OtherTok{\textless{}{-}} \FunctionTok{estimate\_SMP\_params}\NormalTok{(trajectories2, absorbing\_state, dist\_type)}
  
  \CommentTok{\# Calcul des log{-}vraisemblances}
\NormalTok{  logL\_H0 }\OtherTok{\textless{}{-}} \FunctionTok{log\_likelihood}\NormalTok{(mle\_H0, }\FunctionTok{c}\NormalTok{(trajectories1, trajectories2),absorbing\_state, dist\_type)}
  
\NormalTok{  logL\_H1 }\OtherTok{\textless{}{-}} \FunctionTok{log\_likelihood}\NormalTok{(mle\_H1\_t1, trajectories1, absorbing\_state, dist\_type) }\SpecialCharTok{+} \FunctionTok{log\_likelihood}\NormalTok{(mle\_H1\_t2, trajectories2, absorbing\_state, dist\_type) }\CommentTok{\#La log{-}vraisemblance totale est la somme des log{-}vraisemblances des deux ensembles}
  
\NormalTok{  LR }\OtherTok{\textless{}{-}} \FunctionTok{exp}\NormalTok{(logL\_H0 }\SpecialCharTok{{-}}\NormalTok{ logL\_H1)}
  \FunctionTok{return}\NormalTok{(}\FunctionTok{list}\NormalTok{(}\AttributeTok{LR =}\NormalTok{ LR, }\AttributeTok{log\_LR =} \FunctionTok{log}\NormalTok{(LR)))}
\NormalTok{\}}
\end{Highlighting}
\end{Shaded}

\section{3.2 Approche asymptotique
-------------------------------------------------}\label{approche-asymptotique--}

On utilise simplement le fait que -2*ln(LR) converge en distribution
vers khi2.

\begin{Shaded}
\begin{Highlighting}[]
\CommentTok{\#Arguments : LR\_val : Le rapport de vraisemblance entre H0 et H1.}
\CommentTok{\#n\_states : le nombre d\textquotesingle{}états dans le modèle SMP}



\NormalTok{compute\_asymptotic\_pvalue }\OtherTok{\textless{}{-}} \ControlFlowTok{function}\NormalTok{(LR\_val, n\_states, absorbing\_state, dist\_type) \{}
  \CommentTok{\# Calcul des degrés de liberté}
\NormalTok{  k }\OtherTok{\textless{}{-}} \FunctionTok{ifelse}\NormalTok{(dist\_type }\SpecialCharTok{\%in\%} \FunctionTok{c}\NormalTok{(}\StringTok{"gamma"}\NormalTok{, }\StringTok{"weibull"}\NormalTok{), }\DecValTok{2}\NormalTok{, }\DecValTok{1}\NormalTok{) }\CommentTok{\#calcul du nombre de paramètres }
  \ControlFlowTok{if}\NormalTok{ (}\FunctionTok{is.null}\NormalTok{(absorbing\_state)) \{ }\CommentTok{\#le degré de liberté depend de la présence ou non d\textquotesingle{}un état absorbant.}
\NormalTok{    d }\OtherTok{\textless{}{-}}\NormalTok{ n\_states}\SpecialCharTok{\^{}}\DecValTok{2} \SpecialCharTok{{-}}\NormalTok{ n\_states }\SpecialCharTok{{-}} \DecValTok{1} \SpecialCharTok{+}\NormalTok{ k}\SpecialCharTok{*}\NormalTok{n\_states}\SpecialCharTok{*}\NormalTok{(n\_states}\DecValTok{{-}1}\NormalTok{)}
\NormalTok{  \} }\ControlFlowTok{else}\NormalTok{ \{}
\NormalTok{    d }\OtherTok{\textless{}{-}}\NormalTok{ (n\_states}\DecValTok{{-}1}\NormalTok{)}\SpecialCharTok{\^{}}\DecValTok{2} \SpecialCharTok{{-}} \DecValTok{2}\SpecialCharTok{*}\NormalTok{(n\_states}\DecValTok{{-}1}\NormalTok{) }\SpecialCharTok{+}\NormalTok{ k}\SpecialCharTok{*}\NormalTok{(n\_states}\DecValTok{{-}1}\NormalTok{)}\SpecialCharTok{\^{}}\DecValTok{2}
\NormalTok{  \}}
  
  \CommentTok{\# Calcul de la p{-}value}
\NormalTok{  test\_stat }\OtherTok{\textless{}{-}} \SpecialCharTok{{-}}\DecValTok{2}\SpecialCharTok{*}\FunctionTok{log}\NormalTok{(LR\_val)}
\NormalTok{  pval }\OtherTok{\textless{}{-}} \DecValTok{1} \SpecialCharTok{{-}} \FunctionTok{pchisq}\NormalTok{(test\_stat, }\AttributeTok{df =}\NormalTok{ d)}
  \FunctionTok{return}\NormalTok{(pval)}
\NormalTok{\}}
\end{Highlighting}
\end{Shaded}

\newpage

\section{Two-Sample Hypothesis Testing}

We aim to compare the laws of two Semi-Markov Processes (SMP) using two
independent samples of trajectories drawn from two different
populations.

Let and , the parameters associated with the SMPs of each population. We
formulate the hypothesis test:

\begin{equation}
H_0: \theta_1 = \theta_2
\end{equation}

\begin{equation}
H_1: \theta_1 \neq \theta_2
\end{equation}

We assume that we observe independent trajectories from the first
population and from the second. For and , let denote the -th observed
trajectory from population . The complete set of trajectories is denoted
as:

\begin{equation}
S_n = (S_1^1, \dots, S_{n_1}^1, S_{n_1+1}^2, \dots, S_{n_1+n_2}^2).
\end{equation}

\subsection{Global Likelihood Ratio Test}

The test is based on the likelihood ratio , defined as the ratio between
the maximum likelihood under and under :

\begin{equation}
LR =
\frac{\max_{\theta \in \Theta} \prod_{i=1}^{n_1} L(S_i^1; \theta) \times \prod_{j=n_1+1}^{n_1+n_2} L(S_j^2; \theta)}
{\max_{\theta_1 \in \Theta_1, \theta_2 \in \Theta_2} \prod_{i=1}^{n_1} L(S_i^1; \theta_1) \times \prod_{j=n_1+1}^{n_1+n_2} L(S_j^2; \theta_2)}.
\end{equation}

We reject for low values of . More specifically, for a significance
level , we seek the critical value such that:

\begin{equation}
P_{H_0} [LR \leq c_\gamma ] = \gamma.
\end{equation}

One of the main challenges of this test is computing a meaningful
p-value for a given sample.

\subsection{Asymptotic Distribution of }

Under general assumptions, and if and tend to infinity while converges
to a constant , it is established that:

\begin{equation}
\Lambda = -2 \ln (LR) \underset{n \to \infty}{\longrightarrow} \chi^2(d).
\end{equation}

Here, is the difference between the number of parameters estimated under
and under :

\begin{equation}
d =
\begin{cases}
D^2 - D - 1 + |\omega| \times D(D - 1), & \text{without an absorbing state}, \
D^2 - 2D + |\omega| \times (D - 1)^2, & \text{with an absorbing state}.
\end{cases}
\end{equation}\textbackslash end\{equation\}

Where for Gamma and Weibull distributions.

The asymptotic p-value is computed as:

\begin{equation}
p_{asymp} = P(Y \geq \Lambda),
\end{equation}

where and is the observed value of .

\end{document}
